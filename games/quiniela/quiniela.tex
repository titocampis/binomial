\documentclass{article}
\usepackage{graphicx}
\usepackage{amsmath}
\PassOptionsToPackage{svgnames}{xcolor}
\usepackage{tcolorbox}
\usepackage{xcolor}
\usepackage{lipsum}
\usepackage{verbatim}
\tcbuselibrary{skins,breakable}
\usetikzlibrary{shadings,shadows}
\usepackage{float}
\usepackage{hyperref}
\usepackage[a4paper]{geometry}
\usepackage{listings}
\usepackage{titlesec}
\usepackage{amssymb}
\usepackage[T1]{fontenc}
\usepackage{multirow} % for Tables
\usepackage{fancyvrb} % for "\Verb" macro
\VerbatimFootnotes % enable use of \Verb in footnotes
\usepackage{listings}
\lstset{basicstyle=\ttfamily,
  showstringspaces=false,
  commentstyle=\color{green},
  keywordstyle=\color{blue}
}

\setcounter{secnumdepth}{4}
\titleformat{\paragraph}
{\normalfont\normalsize\bfseries}{\theparagraph}{1em}{}
\titlespacing*{\paragraph}
{0pt}{3.25ex plus 1ex minus .2ex}{1.5ex plus .2ex}

\title{\textbf{Quiniela}}
\author{Alejandro Campos}
\date{August, 2023}

\setlength{\parindent}{0ex}
\setlength{\parskip}{6pt}
\geometry{top=2.5cm, bottom=3cm,left=3cm, right=3cm}
\hypersetup{
    colorlinks=true,
    linkcolor=black,
    filecolor=magenta,      
    urlcolor=blue,
}

\definecolor{codegreen}{rgb}{0,0.6,0}
\definecolor{codegray}{rgb}{0.5,0.5,0.5}
\definecolor{codepurple}{rgb}{0.58,0,0.82}
\definecolor{backcolour}{rgb}{0.95,0.95,0.92}

\newenvironment{blocktemplate}[1]{%
    \tcolorbox[beamer,%
    noparskip,breakable,
    colframe=Blue,%
    colbacklower=LimeGreen!75!LightGreen,%
    title=#1]}%
    {\endtcolorbox}

\newenvironment{blocktemplateI}[1]{%
    \tcolorbox[beamer,%
    noparskip,breakable,
    colframe=Violet,%
    colbacklower=Black,%
    title=#1]}%
    {\endtcolorbox}

\newenvironment{blocktemplateII}[1]{%
    \tcolorbox[beamer,%
    noparskip,breakable,
    colframe=Green,%
    colbacklower=LimeGreen!75!LightGreen,%
    title=#1]}%
    {\endtcolorbox}

\newenvironment{blocktemplateIII}[1]{%
    \tcolorbox[beamer,%
    noparskip,breakable,
    ,colframe=Red,%
    colbacklower=LimeGreen!75!LightGreen,%
    title=#1]}%
    {\endtcolorbox}

\newtcolorbox{mybasecolorbox}[1][]{%
  colback=gray!25, colframe=gray!25,
  coltitle=black,
  width=(\linewidth-20pt)}

\newenvironment{codetemplate}[1][]{%
  \mybasecolorbox[#1]
  \itshape
}{%
  \endmybasecolorbox
}

\begin{document}
\maketitle
\newpage
\tableofcontents

%====================================================================================================
\newpage

\section{Introducción}
En este documento analizaremos a fondo las probabilidades de la quiniela, tanto teorica como empiricamente. Además, también analizaremos las variables aleatorias de los premios, ya que no son fijos.

\subsection{Como funciona la quiniela}

La quiniela es un sorteo de loterias y apuestas del estado en el cual debes intentar adivinar el resultado de \textbf{14 partidos} de futbol de una jornada, siendo los \textbf{3 posibles resultados} $=\{1,X,2\}$:
\begin{itemize}
    \item 1 = Victoria Local
    \item x = Empate
    \item 2 = Victoria Visitante
\end{itemize}

\subsection{¿Cuanto cuesta la quiniela?}
La quiniela cuesta \textbf{0,75cts} por \textbf{apuesta simple}. Las apuestas complejas las veremos más adelante

\subsection{¿Como funcionan los premios de la quiniela?}
Los premios de la quiniela dependen principalmente de dos factores:
\begin{itemize}
    \item La recaudación total de esa semana
    \item El numero de personas que han acertado exactamente el mismo numero de partidos.
\end{itemize}

Según el numero de aciertos, el premio será \textbf{a repartir entre los acertantes} y corresponderá a los siguientes porcentajes de la \textbf{recaudación total}:

\begin{enumerate}
    \item Acertar \textbf{10:} 9\%
    \item Acertar \textbf{11:} 7,5\%
    \item Acertar \textbf{12:} 7,5\%
    \item Acertar \textbf{13:} 7,5\%
    \item Acertar \textbf{14:} 16\%
\end{enumerate}

\section{Estudio Probabilidades Teoricas}

\subsection{Apuesta simple suponiendo p = 1/3}

Si suponemos:
\begin{itemize}
    \item $p =$ "Probabilidad de acertar" $= \dfrac{1}{3}$
    \item $q =$ "Probabilidad de fallar" $= 1 - \dfrac{1}{3} = \dfrac{2}{3}$
\end{itemize}

Tenemos una distribución binomial: $X \sim BN(14,\dfrac{1}{3})$.
\\\\
Por tanto, podemos calcular facilmente las probabilidades de acertar:

\begin{itemize}
    \item Acertar \textbf{10} = $\dbinom{14}{10} \cdot \left(\dfrac{1}{3}\right)^{10} \cdot \left(\dfrac{2}{3}\right)^{4} \simeq \dfrac{1}{299} = 0,0035$ 
    \item Acertar \textbf{11} = $\dbinom{14}{11} \cdot \left(\dfrac{1}{3}\right)^{11} \cdot \left(\dfrac{2}{3}\right)^{3} \simeq \dfrac{1}{1.639} = 6,1e-4$ 
    \item Acertar \textbf{12} = $\dbinom{14}{12} \cdot \left(\dfrac{1}{3}\right)^{12} \cdot \left(\dfrac{2}{3}\right)^2 \simeq \dfrac{1}{13.157} = 7,5e-5 $  
    \item Acertar \textbf{13} = $\dbinom{14}{13} \cdot \left(\dfrac{1}{3}\right)^{13} \cdot \dfrac{2}{3} \simeq \dfrac{1}{170.820} = 5,85e-6 $ 
    \item Acertar \textbf{14} = $\left(\dfrac{1}{3}\right)^{14} \simeq \dfrac{1}{4.782.969} = 2,1e-7 $
\end{itemize}

14 y pleno al 15 es algo más complejo, debemos aplicar la siguiente propiedad. Siendo:

\begin{itemize}
    \item $A:$ "Acertar 14 partidos"
    \item $B:$ "Acertar el pleno al 15"
\end{itemize}
$P(A \cap B) = P(A) \cdot P(B | A)$ \\\\
Como son independientes: $P(A \cap B) = P(A) \cdot P(B)$
\begin{itemize}
    \item $P(A) = 2,1e-7$
    \item $P(B) = \dfrac{\text{c.fav}}{\text{c.pos}} = \dfrac{1}{VR_{4}^{2}} = \dfrac{1}{4^2}$
\end{itemize}

Pleno al \textbf{15} = $\dfrac{2,1e-7}{4^2} \simeq \dfrac{1}{76.527.504} = 1.307e-8$

\subsection{Apuesta simple suponiendo p = 27/56}
Sabemos que la probabilidad de acertar un partido no es $\dfrac{1}{3}$, ya que cada partido tiene su propia probabilidad (Barça - Alcorcon no es la misma que Rayo - Cádiz). Por ello, hago un estudio con 112 muestras. De 112 partidos, mis encuestados han acertado 54, lo que nos da $p = \dfrac{27}{56}$

\begin{itemize}
    \item Acertar \textbf{10} = $\dbinom{14}{10} \cdot \left(\dfrac{27}{56}\right)^{10} \cdot \left(1-\dfrac{27}{56}\right)^{4} \simeq \dfrac{1}{20} = 0,049$
    \item Acertar \textbf{11} = $\dbinom{14}{11} \cdot \left(\dfrac{27}{56}\right)^{11} \cdot \left(1-\dfrac{27}{56}\right)^3 \simeq \dfrac{1}{60} = 0,0165$ 
    \item Acertar \textbf{12} = $\dbinom{14}{12} \cdot \left(\dfrac{27}{56}\right)^{12} \cdot \left(1-\dfrac{27}{56}\right)^2 \simeq \dfrac{1}{260} = 0,00385 $  
    \item Acertar \textbf{13} = $\dbinom{14}{13} \cdot \left(\dfrac{27}{56}\right)^{13} \cdot \left(1-\dfrac{27}{56}\right) \simeq \dfrac{1}{1.813} = 5,51e-4 $ 
    \item Acertar \textbf{14} = $\left(\dfrac{27}{56}\right)^{14} \simeq \dfrac{1}{27.261} = 3,67e-05 $
    \item Pleno al \textbf{15} = $\left(\dfrac{27}{56}\right)^{14} \cdot \dfrac{1}
    {4^2}  \simeq \dfrac{1}{436.174} = 2,29e-6
 $
\end{itemize}


\end{document}
