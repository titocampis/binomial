\documentclass{article}
\usepackage{graphicx}
\usepackage{amsmath}
\PassOptionsToPackage{svgnames}{xcolor}
\usepackage{tcolorbox}
\usepackage{xcolor}
\usepackage{lipsum}
\usepackage{verbatim}
\tcbuselibrary{skins,breakable}
\usetikzlibrary{shadings,shadows}
\usepackage{float}
\usepackage{hyperref}
\usepackage[a4paper]{geometry}
\usepackage{listings}
\usepackage{titlesec}
\usepackage{amssymb}
\usepackage[T1]{fontenc}
\usepackage{multirow} % for Tables
\usepackage{fancyvrb} % for "\Verb" macro
\VerbatimFootnotes % enable use of \Verb in footnotes
\usepackage{listings}
\lstset{basicstyle=\ttfamily,
  showstringspaces=false,
  commentstyle=\color{green},
  keywordstyle=\color{blue}
}

\setcounter{secnumdepth}{4}
\titleformat{\paragraph}
{\normalfont\normalsize\bfseries}{\theparagraph}{1em}{}
\titlespacing*{\paragraph}
{0pt}{3.25ex plus 1ex minus .2ex}{1.5ex plus .2ex}

\title{\textbf{Bonoloto}}
\author{Alejandro Campos}
\date{August, 2023}

\setlength{\parindent}{0ex}
\setlength{\parskip}{6pt}
\geometry{top=2.5cm, bottom=3cm,left=3cm, right=3cm}
\hypersetup{
    colorlinks=true,
    linkcolor=black,
    filecolor=magenta,      
    urlcolor=blue,
}

\definecolor{codegreen}{rgb}{0,0.6,0}
\definecolor{codegray}{rgb}{0.5,0.5,0.5}
\definecolor{codepurple}{rgb}{0.58,0,0.82}
\definecolor{backcolour}{rgb}{0.95,0.95,0.92}

\newenvironment{blocktemplate}[1]{%
    \tcolorbox[beamer,%
    noparskip,breakable,
    colframe=Blue,%
    colbacklower=LimeGreen!75!LightGreen,%
    title=#1]}%
    {\endtcolorbox}

\newenvironment{blocktemplateI}[1]{%
    \tcolorbox[beamer,%
    noparskip,breakable,
    colframe=Violet,%
    colbacklower=Black,%
    title=#1]}%
    {\endtcolorbox}

\newenvironment{blocktemplateII}[1]{%
    \tcolorbox[beamer,%
    noparskip,breakable,
    colframe=Green,%
    colbacklower=LimeGreen!75!LightGreen,%
    title=#1]}%
    {\endtcolorbox}

\newenvironment{blocktemplateIII}[1]{%
    \tcolorbox[beamer,%
    noparskip,breakable,
    ,colframe=Red,%
    colbacklower=LimeGreen!75!LightGreen,%
    title=#1]}%
    {\endtcolorbox}

\newtcolorbox{mybasecolorbox}[1][]{%
  colback=gray!25, colframe=gray!25,
  coltitle=black,
  width=(\linewidth-20pt)}

\newenvironment{codetemplate}[1][]{%
  \mybasecolorbox[#1]
  \itshape
}{%
  \endmybasecolorbox
}

\begin{document}
\maketitle
\newpage
\tableofcontents

%====================================================================================================
\newpage

\section{Introducción}
En este documento analizaremos a fondo las probabilidades de l bonoloto, tanto teorica como empiricamente. Además, también analizaremos las variables aleatorias de los premios, ya que no son fijos.

\subsection{Como funciona el bonoloto}
Lotería típica 6/49.

\subsection{¿Cuanto cuesta el bonoloto?}

\subsection{¿Como funcionan los premios del bonoloto?}

\section{Estudio Probabilidades Teóricas}

Por tanto, podemos calcular facilmente las probabilidades de acertar:

\begin{itemize}
    \item Acertar \textbf{3} = $\dfrac{\dbinom{6}{3} \cdot \dbinom{43}{3}}{\dbinom{49}{6}} \simeq \dfrac{1}{57} = 0,0177$ 
    \item Acertar \textbf{4} = $\dfrac{\dbinom{6}{4} \cdot \dbinom{43}{2}}{\dbinom{49}{6}} \simeq \dfrac{1}{1.032} = 9,68e-4$ 
    \item Acertar \textbf{6} = $\dfrac{1}{\dbinom{49}{6}} \simeq \dfrac{1}{13.983.816} = 7,15-8$
\end{itemize}

Para los 5 aciertos tenemos 2 categorias, la categoria de acertar 5 fallando el complementario y la categoría de acertar 5 más el complementario. Aplicaremos la siguiente propiedad. Siendo:

\begin{itemize}
    \item $A:$ "Acertar 5 numeros"
    \item $B:$ "Acertar el complementario"
    \item $\overline{B}:$ "Fallar el complementario"
\end{itemize}
$P(A \cap B) = P(A) \cdot P(B | A)$ \\\\
Como son independientes: $P(A \cap B) = P(A) \cdot P(B)$
\begin{itemize}
    \item $P(A) = \dfrac{\dbinom{6}{5} \cdot \dbinom{43}{1}}{\dbinom{49}{6}} \simeq \dfrac{1}{54.201
} = 1,84e-5$ 
    \item $P(B) = \dfrac{\text{c.fav}}{\text{c.pos}} = \dfrac{1}{43}$
    \item $P(\overline{B}) = \dfrac{\text{c.fav}}{\text{c.pos}} = 1 - \dfrac{1}{43} = \dfrac{42}{43}$
\end{itemize}

Acertar \textbf{5 - c}: $\dfrac{\dbinom{6}{5} \cdot \dbinom{43}{1}}{\dbinom{49}{6}} \cdot \dfrac{42}{43} \simeq \dfrac{1}{55.491} = 1,8e-5$ 

Acertar \textbf{5 + c}: $\dfrac{\dbinom{6}{5} \cdot \dbinom{43}{1}}{\dbinom{49}{6}} \cdot \dfrac{1}{43} \simeq \dfrac{1}{2.330.636} = 4,3e-7$ 



\end{document}
